\documentclass[12pt]{article}

\include{preamble}

\newtoggle{professormode}


\title{MATH 368/621 Fall 2020 Homework \#1}

\author{Frank Palma Gomez} %STUDENTS: write your name here

\iftoggle{professormode}{
\date{Due by email 11:59PM Saturday, September 12, 2020 \\ \vspace{0.5cm} \small (this document last updated \today ~at \currenttime)}
}

\renewcommand{\abstractname}{Instructions and Philosophy}

\begin{document}
\maketitle

\iftoggle{professormode}{
\begin{abstract}
The path to success in this class is to do many problems. Unlike other courses, exclusively doing reading(s) will not help. Coming to lecture is akin to watching workout videos; thinking about and solving problems on your own is the actual ``working out.''  Feel free to \qu{work out} with others; \textbf{I want you to work on this in groups.}

Reading is still \textit{required}. For this homework set, review Math 241 concerning random variables, support, parameter space, PMF's, CDF's, bernoulli, binomial, geometric. Then read on your own about the negative binomial, convolutions and the multinomial distribution.

The problems below are color coded: \ingreen{green} problems are considered \textit{easy} and marked \qu{[easy]}; \inorange{yellow} problems are considered \textit{intermediate} and marked \qu{[harder]}, \inred{red} problems are considered \textit{difficult} and marked \qu{[difficult]} and \inpurple{purple} problems are extra credit. The \textit{easy} problems are intended to be ``giveaways'' if you went to class. Do as much as you can of the others; I expect you to at least attempt the \textit{difficult} problems. \qu{[MA]} are for those registered for 621 and extra credit otherwise.

This homework is worth 100 points but the point distribution will not be determined until after the due date. See syllabus for the policy on late homework.

Up to 7 points are given as a bonus if the homework is typed using \LaTeX. Links to instaling \LaTeX~and program for compiling \LaTeX~is found on the syllabus. You are encouraged to use \url{overleaf.com}. If you are handing in homework this way, read the comments in the code; there are two lines to comment out and you should replace my name with yours and write your section. The easiest way to use overleaf is to copy the raw text from hwxx.tex and preamble.tex into two new overleaf tex files with the same name. If you are asked to make drawings, you can take a picture of your handwritten drawing and insert them as figures or leave space using the \qu{$\backslash$vspace} command and draw them in after printing or attach them stapled.

The document is available with spaces for you to write your answers. If not using \LaTeX, print this document and write in your answers. I do not accept homeworks which are \textit{not} on this printout. Keep this first page printed for your records.

\end{abstract}

\thispagestyle{empty}
\vspace{1cm}
NAME: \line(1,0){380}
\clearpage
}

\problem{These exercises give you practice with sums and indicator functions.}

\begin{enumerate}

\easysubproblem{Expand and simplify as much as you can: $\sum_{x \in \reals} \indic{x = 17}$.}\spc{0.5}

\begin{align*}
    \sum_{x \in \reals} \indic{x = 17} &= 1
\end{align*}

\easysubproblem{Expand and simplify as much as you can: $\sum_{x \in \reals} c \indic{x = 17}$ where $c \in \reals$ is a constant.}\spc{0.5}

\begin{align*}
    \sum_{x \in \reals} c \indic{x = 17} &= c 
\end{align*}

\easysubproblem{Expand and simplify as much as you can: $\sum_{x \in \reals} \indic{x \in \braces{1, 2, 3}}$.}\spc{0.5}

\begin{align*}
    \sum_{x \in \reals} \indic{x \in \braces{1, 2, 3}} &= 3
\end{align*}

\easysubproblem{Expand and simplify as much as you can: $\sum_{x \in \reals} x\indic{x \in \braces{1, 2, 3}}$.}\spc{0.5}

\begin{align*}
    \sum_{x \in \reals} x\indic{x \in \braces{1, 2, 3}} &= 1(1) + 2(1) + 3(1) + 4(0) + 5(0) + \ldots \\ &= 6 
\end{align*}

\easysubproblem{Expand and simplify as much as you can: $\sum_{x \in \naturals_0} x^{\indic{x \in \braces{1, 2, 3}}}$.}\spc{0.5}

\begin{align*}
    \sum_{x \in \naturals_0} x^{\indic{x \in \braces{1, 2, 3}}} &= 1^1 + 2^1 + 3^1 + 4^0 + \ldots \\ &= 6 + \infty \\ &= \infty
\end{align*}

\easysubproblem{Expand and simplify as much as you can: $\prod_{x \in \reals} \indic{x \in \braces{1, 2, 3}}$.}\spc{0.5}

\begin{align*}
    \prod_{x \in \reals} \indic{x \in \braces{1, 2, 3}} &= 1 * 1 * 1 * 0 * \dots \\ &= 0
\end{align*}


\easysubproblem{Expand and simplify as much as you can: $\sum_{x \in \reals} \indic{x \in \braces{1, 2, 3}}\indic{x \in \braces{4, 5, 6}}$.}\spc{0.5}

\begin{align*}
    \sum_{x \in \reals} \indic{x \in \braces{1, 2, 3}}\indic{x \in \braces{4, 5, 6}} &= (1)(0) + (1)(0) + (1)(0) + (0)(1) + (0)(1) + (0)(1) + \ldots \\ &= 0
\end{align*} 

\intermediatesubproblem{Expand and simplify as much as you can: $\sum_{x \in \reals} c \indic{x \in \braces{1, 2, \ldots, t}}$ where $c \in \reals$ is a constant and $t \in \naturals$ is a constant.}\spc{0.5}

\begin{align*}
    \sum_{x \in \reals} c \indic{x \in \braces{1, 2, \ldots, t}} &= c \sum_{i=0}^{t} i \\ &= tc 
\end{align*}

\intermediatesubproblem{Expand and simplify as much as you can: $\sum_{x \in \reals} t \indic{x \in \braces{1, 2, \ldots, t}}$ where $c \in \reals$ is a constant and $t \in \naturals$ is a constant.}\spc{0.5}

\begin{align*}
    \sum_{x \in \reals} t \indic{x \in \braces{1, 2, \ldots, t}} &= t \sum_{i=1}^{t} i \\ &= t^2
\end{align*}
\intermediatesubproblem{Expand and simplify as much as you can: $\sum_{x \in \reals} x \indic{x \in \braces{1, 2, \ldots, t}}$ where $c \in \reals$ is a constant and $t \in \naturals$ is a constant.}\spc{0.5}

\begin{align*}
    \sum_{x \in \reals} x \indic{x \in \braces{1, 2, \ldots, t}} &= \sum_{i = 1}^{t}i \\ &= \frac{(t-1)(t)}{2} 
\end{align*}

\intermediatesubproblem{Expand and simplify as much as you can: $\sum_{x \in \reals} \oneover{x!} \indic{x \in \naturals}$.}\spc{0.5}

\begin{align*}
    \sum_{x \in \reals} \oneover{x!} \indic{x \in \naturals} &= \sum_{x \in \naturals} \oneover{x!} \\ &= \exp{1}
\end{align*} 


\intermediatesubproblem{Prove $\expe{\indic{X \in A}} = \prob{X \in A}$.}\spc{2}


\end{enumerate}

\problem{These exercises review convolutions.}


\begin{enumerate}

\easysubproblem{Is a JMF a type of PMF or PMF a type of JMF? Explain.}\spc{2}

\begin{align*}
    \text{A JMF is a type of PMF because a JMF is derived from 2 or more PMF's}
\end{align*}

\easysubproblem{Let $X_1, X_2 \iid \bernoulli{p}$. Find the PMF of the sum of $T = X_1 + X_2$ using the appropriate discrete convolution formula that would make the problem easiest.}\spc{4}

\begin{align*}
    \prob{t} &= \sum_{x \in \reals} {1 \choose x}p^x(1-p)^{1-x} {1 \choose t-x}p^{t-x}(1-p)^{1-t-x} \\
    &= p^t(1-p)^{2-t} \sum_{i \in \reals} {1 \choose x}{1 \choose t-x} \\
    &= p^t(1-p)^{2-t}\Big({1 \choose t}{1 \choose t-1}\Big) \\
    &= {2 \choose t}p^{t}(1-p)^{2-t}
\end{align*}

\easysubproblem{Let $X_1 \sim \bernoulli{p_1}$ independent of $X_2 \sim \bernoulli{p_2}$. Find the JMF of for $X_1, X_2$. Denote it using a 2 $\times$ 2 grid or the piecewise function notation.}\spc{6}

\begin{align*}
    \begin{cases}
        1 \withprob (p_1)(p_2) \\ 
        0 \withprob (1-p_1)(1-p_2)
    \end{cases}
\end{align*}


\hardsubproblem{Let 

\beqn
X_1 \sim \begin{cases}
3 \withprob 0.3 \\
6 \withprob 0.7
\end{cases} \quad \text{independent of} \quad
%
X_2 \sim \begin{cases}
4 \withprob 0.4 \\
8 \withprob 0.6
\end{cases} 
\eeqn

Find the PMF of $T = X_1 + X_2$ using a convolution. Denote it using the piecewise function notation.}\spc{6}


\hardsubproblem{Prove the PMF of a binomial inductively using convolutions on the sequence of r.v.'s $\Xoneton \iid \bernoulli{p}$. You will need to use Pascal's Triangle combinatorial identity we employed in class.}\spc{11}

\begin{align*}
    \text{Let $T_n = X_1 + X_2 + \ldots + X_n$ and $T_n = X_n + T_{n-1}$} 
\end{align*}

\begin{align*}
    \prob{t} &= \sum_{x \in \braces{0, 1}} p^x(1-p)^{1-x} {n-1 \choose t-x}p^{t-x}(1-p)^{n-1-t+x} \\
    &= \sum_{x \in \braces{0, 1}} p^t(1-p)^{n-t}{n-1 \choose t-x} \\
    &= p^t(1-p)^{n-t} \sum_{x \in \braces{0, 1}}{n-1 \choose t-x} \\ 
    &= p^t(1-p)^{n-t} \Big( {n-1 \choose t} + {n-1 \choose t-1}\Big) \\ 
    &= p^t(1-p)^{n-t} {n \choose t}
\end{align*}

\hardsubproblem{[MA] Prove the PMF of a negative binomial inductively using convolutions on the sequence of r.v.'s $\Xoneton \iid \geometric{p}$. You will need to use the \qu{hockey stick identity} \href{https://en.wikipedia.org/wiki/Hockey-stick_identity}{[click here]}.}\spc{11}

\hardsubproblem{Let $X_1 \sim \binomial{n_1}{p}$ independent of $X_2 \sim \binomial{n_2}{p}$. Find the PMF of the sum of $T = X_1 + X_2$ using a convolution.}\spc{13}

\begin{align*}
    \prob{t} &= \sum_{x \in \reals} {n_1 \choose x}p^x(1-p)^{n_1-x}{n_2 \choose t-x}p^{t-x}(1-p)^{n_2-t+x}\indic{t-x \in \support{X_1}} \\ 
    &= p^t(1-p)^{n_1+n_2-t}\sum_{x \in \braces{0, \ldots, t}}{n_1 \choose x}{n_2 \choose t-x} \\
    &= p^t(1-p)^{n_1+n_2-t}{n_1 + n_2 \choose t}
\end{align*}

\easysubproblem{Prove the PMF of $X \sim \poisson{\lambda}$ using the limit as $n \rightarrow \infty$ and let $p = \overn{\lambda}$.}\spc{9}


\hardsubproblem{Let $X_1 \sim \poisson{\lambda_1}$ independent of $X_2 \sim \poisson{\lambda_2}$. Find the PMF of the sum of $T = X_1 + X_2$ using a convolution.}\spc{12}



\end{enumerate}


\problem{These exercises introduce probabilities of conditional subsets of the supports of multiple r.v.'s.}


\begin{enumerate}

\hardsubproblem{Let $X \sim \geometric{p_x}$ independent of $Y \sim \geometric{p_y}$. Find $\prob{X > Y}$ using the method we did in class.}\spc{10}


\easysubproblem{[MA] Prove this a different way by finding $\prob{X = Y}$ and then using the law of total probability.}\spc{10}

\easysubproblem{[MA] As both $p_x$ and $p_y$ are reduced to zero, but $r = \frac{p_x}{p_y}$, what is the asymptotic probability you found in (a)?}\spc{10}

\hardsubproblem{Let $X \sim \poisson{\lambda}$ independent of $Y \sim \poisson{\lambda}$. Find an expression for $\prob{X > Y}$ \emph{as best as you are able to answer}. Part of this exercise is identifying where you cannot go any further.}\spc{9}


\end{enumerate}


\problem{These exercises will introduce the Multinomial distribution.}


\begin{enumerate}

\easysubproblem{If $\X \sim \multinomial{n}{\p}$ where $\dime{\X} = k$, what is the parameter space for both $n$ and $\p$?}\spc{2}

\easysubproblem{If $\X \sim \multinomial{n}{\p}$ where $\dime{\X} = k$, what is the $\support{\X}$?}\spc{2}

\easysubproblem{If $\X \sim \multinomial{n}{\p}$ where $\dime{\X} = k$, what is $\dime{\p}$?}\spc{-0.5}

\easysubproblem{If $\X \sim \multinomial{n}{\p}$ where $\dime{\X} = 2$, express $p_2$ as a function of $p_1$.}\spc{2}

\easysubproblem{If $\X \sim \multinomial{n}{\p}$ where $\dime{\X} = 2$, how are both $X_1$ and $X_2$ distributed?}\spc{1}

\easysubproblem{If $\X \sim \multinomial{n}{\p}$ and $n= 10$ and $\dime{\X} = 7$ as a column vector, give an example value of $\x$, a realization of the r.v. $\X$.}\spc{2}

\easysubproblem{If $\X \sim \multinomial{9}{\bracks{0.1~0.2~0.7}^\top}$, find $\prob{\X = \bracks{3~2~4}^\top}$ to the nearest two decimal places.}\spc{2}


%\intermediatesubproblem{If $\X \sim \multinomial{n}{\p}$ and $n= 10$ and $\p = \bracks{0.2, 0.8}^\top$, find $\muvec := \expe{\X}$.}\spc{2}


\hardsubproblem{[MA] If $\X_1 \sim \multinomial{n}{\p}$ and independently $\X_2 \sim \multinomial{n}{\p}$ where $\dime{\X_1} = \dime{\X_2} = k$. Find the JMF of $\T_2 = \X_1 + \X_2$ from the definition of convolution. This looks harder than it is! First, use the definition of convolution and factor out the terms that are not a function of $x_1, \ldots, x_K$. Finally, use Theorem 1 in this paper: \href{http://www.lrecits.usthb.dz/1.3.pdf}{[click here]} for the summation.}\spc{9}

\end{enumerate}


\end{document}
